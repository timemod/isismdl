\documentclass[english,memorandum,old,sectionpage]{cpbart}
%\documentclass[english,memorandum,old]{cpbart}
\usepackage[english]{babel}
\usepackage{cpbfonts}
\usepackage{amsmath}
\usepackage[psamsfonts]{amsfonts}
%
\newcommand{\Ina}{$\qquad$}
\newcommand{\Inb}{\Ina\Ina}
\newcommand{\Inc}{\Ina\Inb}
%
%
% margins for ruled figure
%
\makeatletter
\def\Rf@margininit{%
\setlength\leftmargini{\unit@indent}%
\setlength\leftmarginii{\unit@indent}%
\setlength\leftmarginiii{\unit@indent}%
\setlength\leftmarginiv{\unit@indent}%
\setlength\leftmarginv{\unit@indent}%
\setlength\leftmarginvi{1em}%
\setlength\leftmargin{\leftmargini}%
\setlength\mathindent{\leftmargini}%
}
%
% figure with horizontal rule before and after
%
\newenvironment{Rfigure}%
{\begin{figure}\rule{\linewidth}{0.1mm}}%
{\rule{\linewidth}{0.1mm}\end{figure}}%
%
% for large algorithm
%
\newcommand{\Algofont}{}%{\footnotesize}
\makeatletter
%\newenvironment{progalg}{\addvspace{-0.75\baselineskip}\begin{boxit}\parindent\z@\obeylines\Algofont}%
%{\end{boxit}}
\newenvironment{progalg}{\Rf@margininit\parindent\z@\obeylines\Algofont}%
{}
\makeatother
%
%
\newcommand{\Bfrac}[2]{\frac{\displaystyle #1}{\displaystyle #2}}
\newcommand{\extravspace}{\vspace{.5\baselineskip}}
%
\newcommand{\V}[1]{\mathit{#1}}
\newcommand{\Rbb}{\mathbb{R}}
%
% for keywords in algorithm
%
\newcommand*{\Pkword}[1]{\textbf{#1}}
%
\newcommand*{\Wlin}[1]{D #1 + w - h(#1)}
\newcommand*{\ubar}{\bar{u}}
\newcommand*{\Dbar}{\bar{D}}
%
%
\begin{document}

\documentname{A Newton algorithm for solving an underdetermined system of equations}
\CPBafdeling{OMD}
\Notitienummer{2006/OMD/1}
\author{B.H. Hasselman}
\title{A Newton algorithm for solving an underdetermined system of equations}
\date{\today}
\begin{abstract}
When making short term macro-economic forecasts, there is often
no clear transition from a period with a known and given history
to a period with only forecasts.
Often some information for a few economic variables such as
unemployment, inflation and interest rates are already available for the first
few quarters of a forecasting period.

When a macro-economic model is used as a tool for forecasting, it would be very
useful if any available partial  information on some economic variables
could be used to make a conditional forecast.

This paper describes the method to achieve this that the CPB has implemented
in its proprietary software package for econometric modelling.
It also provides technical background information for the computer implementation.
\Bpar
I wish to thank Arie ten Cate, Debby Lanser, Hans Lunsing and Bert Smid for their
comments and suggestions.
\end{abstract}
%
\maketitle
%
\tableofcontents

\section{Introduction}
%\footnote{This memorandum is a modified and extended version of a
%memorandum with the same title published in 1997.}
When an econometric model is used for forecasting, it may be beneficial or
required to use preliminary observations in the forecasting period
on certain economic quantities
such as unemployment or inflation. These observations provide partial
additional information for the forecast and one would like to make a forecast with
the model conditional on these observations.
\cite*{sdbfit} formulate the problem as follows

\begin{quotation}\itshape
Economic time-series can be predicted by the solution, period after period, of
the equations of a model that uses exogenous projections and past endogenous
variables as inputs.

For the more distant past, statistical information about all variables will be
available. If these data are substituted in the model, the behavioural equations
show non-zero residuals.

In the more recent periods, still in the past, a limited set of variables has
already been observed. To produce an up-to-date forecast it is important to
incorporate this recent statistical information.
\end{quotation}
%
%
The most natural way to do this is to find values for error terms in the
model such that the forecasted values for the observed
quantities are equal to the observations.

Sometimes it may be necessary for an econometric model
to yield predetermined outcomes for certain endogenous variables.
Examples are

\begin{itemize}
\item making an annual model reproduce forecasts made with a quarterly model.
\item making sectoral forecasts conditional on macroeconomic forecasts,
which serve as the recent observations.
\item forcing a model to generate a ``desired outcome'', for example
a required medium term growth of GDP.
\item one may wish to investigate the behaviour of a model for
a zero government budget deficit.
\end{itemize}
%
In all these cases the desired result may be achieved
by manipulating  error terms to force the model to yield
the required outcome.

Usually a model has many more relevant error terms than
observations.
Therefore generally speaking there are infinitely many solutions
for the error terms.
From a statistical point of view it seems natural to require that the
vector of error terms has minimum Euclidean norm.
The  problem can be formulated mathematically as follows
\begin{gather*}
\min _u \quad u^T u    \\
\mbox{subject to} \quad w = h(u)
\end{gather*}
with $u \in \Rbb^n$, $w \in \Rbb^m$, $h:\Rbb^n \rightarrow \Rbb^m$
and $m \le n$.
The vector $w$ represents the
targets for some endogenous variables of the underlying system of equations.
The vector $u$ represents the error terms to be used for achieving the targets.
The vector valued function $h$ represents the -- usually implicit --
reduced form equations relating the targeted endogenous variables to the residuals.
The error terms in this formulation of the problem
are equal to the model error terms scaled with a positive factor
such as a standard deviation or a root-mean-squared-error.

In the following sections first the mathematics of the problem are discussed.
Then the computation of the jacobian matrix is discussed and
the algorithm is presented in the form of pseudo code;
it is based on \cite{sanfit}.
Finally, some cautionary remarks are made and
possible extensions are discussed.

\section{Mathematics}
As stated in the introduction the task is to solve
the following minimisation program
\begin{subequations}\label{eqn:min}
\begin{gather}
\min _u \quad u^T u    \\
\mbox{subject to} \quad w = h(u)
\end{gather}
\end{subequations}
with $u \in \Rbb^n$, $w \in \Rbb^m$, $h:\Rbb^n \rightarrow \Rbb^m$
and $m \le n$.
When $m < n$ the system of equations $w = h(u)$ is underdetermined and
the minimisation is equivalent to finding the minimum norm solution of
\begin{equation}\label{eqn:orgweqh}
w = h(u)
\end{equation}
We assume that a solution exists.

\subsection*{Lagrange method}
Using the Lagrange method the problem can be reformulated as
\begin{equation}\label{eqn:lagrangefull}
\min_{u, \lambda} \tfrac{1}{2} u^T u + \lambda \bigl(w - h(u)\bigr)
\end{equation}
with $\lambda \in \Rbb^m$.
%
The first order necessary conditions are
\begin{subequations}
\begin{align}
u &= D^T \lambda \label{eqn:ueqdtl} \\
w &=h(u)  \label{eqn:weqh}
\end{align}
\end{subequations}
%
where $D \in \Rbb^{m \times n}$.
The matrix $D$ is the jacobian of $h$ with
$D_{ij} = \partial h_i / \partial u_j$. The matrix $D$ is assumed to
have full row rank ($\operatorname{rank}(D)=m$).

In order to arrive at an explicit expression for $u$ it is necessary
to linearise equation \eqref{eqn:weqh}
in a neighbourhood of $\ubar$
\begin{equation} \label{eqn:uw}
w = h(\ubar) + D \bigl( u - \ubar \bigr)
\end{equation}
%
Rearrange this equation to read
\begin{equation} \label{eqn:uwa}
D u =  w - h(\ubar) + D\ubar
\end{equation}
%
Premultiply equation \eqref{eqn:ueqdtl} with $D$ to get
\begin{equation}
D u = (D D^T) \lambda
\end{equation}
%
Substitute this into equation \eqref{eqn:uwa} and rearrange to get
the following expression for $\lambda$
\begin{equation}
\lambda = (D D^T)^{-1} \bigl( \Wlin{\ubar} \bigr)
\end{equation}
%
Finally, substitute this into equation \eqref{eqn:ueqdtl} to obtain
the following expression for $u$
\begin{equation}
u = D^T (D D^T)^{-1} \bigl( \Wlin{\ubar} \bigr)
\end{equation}

\subsection*{Linear equation method}

Exactly the same -- approximate -- solution to the original
optimisation problem can be obtained by first linearising
equation~\eqref{eqn:orgweqh}.
But then application of Lagrange's method  in equation \eqref{eqn:lagrangefull}
is equivalent to finding the minimum norm solution of the system of
linear equations
%
\begin{equation}
D u =  D \ubar+ w - h(\ubar)
\end{equation}
%
where $D$ is the jacobian of $h(\ubar)$.
The minimum norm solution for this system of linear equations is
(see \cite*{nlaopt} and \cite*{clineplemmons})
%
\begin{equation}\label{eqn:slnfrank}
u = D^T (D D^T)^{-1} \bigl( \Wlin{\ubar} \bigr)
\end{equation}
or more generally
This can also be written as
\begin{equation}\label{eqn:slnmp1}
u = D^{+} \bigl( \Wlin{\ubar} \bigr)
\end{equation}
where $D^{+}$ is the Moore-Penrose inverse of $D$.%
\footnote{This equation also holds when $D$ does not have full row rank.
See \cite{nlaopt} and \cite{matcomp}.}

If $\ubar$ was calculated in a similar manner with a matrix
$\Dbar$ identically equal to the current matrix $D$,
then  $\ubar$ is the minimum norm solution of a system $D u = b$.
That implies $\ubar = D^+ b$ and $D^+ D \ubar = D^+ b = \ubar$.
Then equation~\eqref{eqn:slnmp1} can be simplified to
\begin{equation}\label{eqn:slnmp2}
u = \ubar+ D^+ \bigl(w - h(\ubar)\bigr)
\end{equation}
%
When the jacobian $D$  is evaluated at every iteration of a
full Newton-like algorithm then  equation~\eqref{eqn:slnmp1}
should be used to obtain a minimum norm residual vector at each iteration.
However, when the jacobian $D$  is not evaluated at every iteration
-- for whatever reason --
then  equation~\eqref{eqn:slnmp2} may be used
whenever the jacobian is retained across iterations.

It is inadvisable to compute $D^+$ directly using $D D^T$
since this can lead to loss of numerical accuracy. If
$\kappa(D)$ is the condition number of matrix $D$ then the condition number
of $D D^T$ will be $\kappa(D)^2$.
This implies that even a moderately ill-conditioned $D$ can lead
to numerical inaccuracy.\footnote{See \cite{clineplemmons}.}

When $D$ has full row rank, $D^+$ can be computed from the
$\V{QR}$ factorisation of $D^T$%
\begin{displaymath}
D^T = QR \quad \Rightarrow \quad D^+ = Q R^{-T}
\end{displaymath}
where $Q\in \Rbb^{n \times m}$ has orthonormal columns\footnote{Therefore $Q^TQ=I$.}
and $R \in \Rbb^{m \times m}$ is upper triangular.%
\footnote{where $R^{-T}=(R^T)^{-1}$.}
Hence $u = D^+ b$ can be computed with the following two steps

\begin{enumerate}
\item solve $R^T z = b$
\item set $u = Q z$
\end{enumerate}
This procedure is numerically stable.
See \cite{matcomp} and \cite{nlaopt} for more details.

In numerical libraries a $\V{QR}$ factorisation will often overwrite
the original matrix. As a result calculation of $D \ubar$ would
have to be done using $R^T Q^T$, which could introduce numerical
inaccuracies. These may be avoided by using equation~\eqref{eqn:slnmp2}
whenever appropriate.

%If allowance must be made for loss of rank of $D$, then other methods
%for determining $D^+$ have to be used such as the \textsl{QR} factorisation with
%column pivoting or even the \textsl{SVD} decomposition.

\section{Computing the jacobian $D$}\label{sec:calcD}
Column $j$ of the fit jacobian matrix $D$ contains the derivatives
of the target endogenous variables with respect to the residual $u_j$.
If the matrix can only be computed numerically, the following
steps are taken to calculate a column of the $D$ matrix.
We assume a baseline solution has been calculated.
Let $u$ be the vector of \emph{scaled} residuals and
let $\ubar$ be the vector of current values of
the residuals. Let $q$ represent the vector of endogenous
variables which are being targeted.
And let $\bar{q}$ be the vector of current values of the targeted
endogenous variables.

The method for computing the successive columns of the jacobian $D$
proceeds as follows

\begin{enumerate}
\item Change a residual with a small perturbation ($\delta$)
\begin{displaymath}
u_j = \ubar_j + \delta
\end{displaymath}

\item Solve the model to a specified convergence level
\item Calculate column $j$ of the $D$ matrix
\begin{displaymath}
D_{.,j} = (q - \bar{q})/\delta
\end{displaymath}
Remember that $D$ is the jacobian with respect to the \emph{scaled} residuals.
\end{enumerate}
%
The constant $\delta$ is set to $0.1$, which has proven to be adequate.

In our computer implementation\footnote{which uses the feedback method
for solving normalised systems of equations; see \cite{dongallo}.}
the steps for solving the model in this case are as follows

\begin{enumerate}
\item Make one pass through the model equations
\item Compute the Newton step for the feedback variables
\item Adjust the feedback variables
\item Make a second pass through the model equations.
\end{enumerate}
%
The model is not solved to a specified convergence tolerance but
two Newton iterations are taken.
It has proven to be sufficient for achieving sensible outcomes.


\newpage

\section{Algorithm}

An algorithm for computing the minimum norm solution of our nonlinear
system of equations should allow for retaining the Jacobian $D$
between iterations, to avoid possibly very expensive calculations.
A norm on $w - h(u)$ is used for testing
convergence. Define $Z=w-h(u)$.
When the norm on $Z$ does not show a certain minimal reduction and the jacobian
matrix is up to date, the algorithm stops due to lack of progress.
Finally, if convergence is slow as measured by insufficient relative
reduction in the norm on $Z$, the Jacobian is recalculated.
The algorithm in pseudo code is given in figure~\ref{figalg}.
The condition of the jacobian $D$ with full row rank is estimated
with the \textsc{Linpack} condition estimator;
if the estimated \emph{inverse condition number} of $D$ is
less than the square root of the machine precision, then
the algorithm stops; the jacobian is too ill-conditioned for
the algorithm to proceed sensibly.

The norm on $Z$ used in the algorithm for testing
convergence and progress may be any norm (Euclidian, infinity or scaled
infinity).
The constant $\eta_w$ should be small, typically $10^{-3}$, but must
be larger than the accuracy with which
$Z$ has been computed. For example, if $Z$ has been computed using a
convergence test of $10^{-4}$ then $\eta_w$ should be larger than $10^{-4}$.

The fixed constant $\gamma_p \in (0,1)$
is intended to ensure sufficient decrease;
it should be close to $1$. In the actual computer implementation
it has been set to $0.95$; experiments in the distant past
have shown that a larger value less than $1.$ was never
successful as remedy for any numerical difficulties.

Finally the constant $\eta_s \in (0,\gamma_p)$ determines when to
recalculate the jacobian $D$ (the default is $0.5$).


\begin{Rfigure}
\caption{Fit algorithm}\label{figalg}
\input{fitalg}
\end{Rfigure}

\clearpage
\section{Remarks}
The algorithm has been tested on

\begin{itemize}

\item a historic medium-size (approximately 1500 equations) quarterly
forecasting macro economic model using 27 residuals,
6 observations on the first forecasting quarter and 3 observations
in the following three forecasting quarters. The algorithm
needs two iterations in each quarter to converge.

\item a current quarterly forecasting model with approximately 2500 equations,
using 22 residuals
and 22 observations and targets for the first quarter of the
forecasting period and 8 residuals for the remaining 11 quarters.
Here too, the algorithm needs two iterations to converge.

\item the current annual medium term macroeconomic forecasting model,
having approximately 2500 equations,
using 22 residuals and 2 targets (unemployment and enterprise production).
The algorithm needs 1 iteration to converge.
\end{itemize}
%
The procedure has also been used with success with much larger models and
using large number of residuals and observations or targets.

These tests and others not mentioned explicitly give
rise to the following remarks

\begin{enumerate}
\item when the functions relating targets and residuals
      -- the function $h$ in equation~\ref{eqn:orgweqh} --
      are implicitly determined by an iterative process
      as would normally be the case for an econometric model, great care
      must be taken in specifying
      the convergence threshold. It should not be smaller than the
      convergence threshold used in solving the underlying system and
      may even have to be considerably larger.

\item no global strategy such as a line search has been incorporated in
      the algorithm. Experiments with a backtracking line search did not
      yield satisfactory results, which may also have been caused by the
      method of solving the underlying system. Numerical problems
      could always be avoided by changing
      the specification of the underlying model.

\item since evaluating the Jacobian can be a very expensive process
      it may be worthwhile to investigate secant like adjustments to the
      Jacobian analogous to Broyden's method for square nonlinear
      equations.
\end{enumerate}

\clearpage
\bibliography{../newt}
\bibliographystyle{cpblit}

\end{document}

